\documentclass[12pt,fleqn]{article}

\usepackage[utf8]{inputenc}
%\usepackage[latin1]{inputenc}
\usepackage{rotating}
\usepackage{booktabs}
\usepackage{empheq}
\usepackage{geometry}
\usepackage{parskip}
\usepackage{amsmath}
\usepackage{amssymb}
\usepackage{siunitx}
\usepackage{inconsolata}
\usepackage{graphicx}

\DeclareMathOperator{\gam}{Gamma}

\title{\textit{Devoir 2}\\Évaluation statistique des erreurs de réplication par
  couverture delta et couverture optimale} 

\date{HEC Montréal\\[4em]Le 29 octobre 2015} 

\author{Thierry \textsc{Bazier-Matte} \and Yann \textsc{Foucault} \and Annalaura
  \textsc{Capuano} \and Mohammed \textsc{Huzeyen} \and Qing \textsc{Yu}}

\begin{document}

\maketitle

\section*{Introduction}

L'émission d'options de vente sur des actifs financiers comme des actions est rendu
extrêmement fréquents, et de grosses sommes sont allouées dans ce type de commerce. il est
donc essentiel pour l'émetteur d'une telle option de vente de considérer de quelle facon
il peut répliquer la valeur de l'option de manière prévisible, afin de pouvoir essuyer le
payoff possiblement élevé qu'il devra payer à l'acheteur si l'option finit dans la
monnaie.

Nous ferons donc une analyse statistique des pertes possibles suivant l'émission d'une
telle option de vente. Nous étudierons particulièrement la méthode delta, basée sur le
mouvement brownien géométrique des prix, ainsi que la méthode optimale elle aussi concue
avec la théorie Black Scholes. En fait, chacune des deux méthodes, dans un contexte Black
Scholes doivent tendre vers une erreur nulle lorsqu'on considère un rebalancement
infini. Ce n'est bien sûr pas le cas en réalité et nous tiendrons donc compte de la
fréquence des rebalancements du portefeuille de réplication. Nous allons également
considérer les cas où l'option commence à, hors, ou dans la monnaie. 

Toutefois, il est connu que le modèle Black Scholes est trop simpliste pour représenter
fidèlement la réalité, et ainsi nous allons également considérer un modèle alternatif
variance gamma afin de déterminer comment se comporte les stratégies de couverture dans un
contexte différent de celui dans lequel elles ont été créées. 

\section*{Question 1}

Nous avons codé une fonction de type 
\begin{equation*}
  \verb+Fonction [V] = hedging (X, K, r, T, mu, sigma, put)+
\end{equation*}
composée des paramètres suivants:
\begin{itemize}
\item \verb+X+: Matrice de prix, où les rangées représentent les observations et les colonnes
  les simulations;
\item \verb+K+: Prix de levée de l'option;
\item \verb+r+: Taux annuel sans risque;
\item \verb+T+: \'Echeance de l'option;
\item \verb+sigma+: Volatilité du sous-jacent;
\item \verb+put+: booléen numérique représentant le type d'option à couvrir.
\end{itemize}

Voici le code de la fonction en question:
\small{
\begin{equation*}
  \left.
    \begin{array}{l}
       \verb|function V = hedging(S,K,r,T,mu,sigma,put)|\\
      \verb|[n,~] = size(S);                        % S: nxN|\\
      \verb|beta = diag(exp(-r*linspace(0,T,n)));   % discount factor diagonal|\\
      \verb|discountS = beta*S;                     % Discounted prices.|\\
      \verb|Delta = discountS(2:end,:) - discountS(1:end-1,:);|\\
      \verb|T = linspace(T,0,n)';|\\
      \verb|[phiCall, phiPut] = blsdeltam(S(1:end-1,:),K,r,T(1:end-1),sigma);|\\
      \verb|[V0Call, V0Put] = blspricem(S(1),K,r,T(1),sigma);|\\
      \verb||\\
      \verb|if ~put|\\
      \verb|    V = V0Call + sum(phiCall.*Delta,1)';|\\
      \verb|else|\\
      \verb|    V = V0Put + sum(phiPut.*Delta,1)';|\\
      \verb|end|\\
      \verb|end|\\
    \end{array}\right.
\end{equation*}}

Cette fonction retourne la valeur actualisée du portefeuille de couverture selon chacun
des scénarios présents dans la matrice \verb+X+ en employant une méthode de couverture
delta. 

La valeur du portefeuille est alors donnée par
\begin{equation*}
  V = V_0 + \sum_{i=1}^n \phi_k \Delta_k,
\end{equation*}
ce qui se traduit dans le contexte algorithmiquement par 
\begin{equation*}
  \verb!V = V0Put + sum(phiPut.*Delta,1)';!
\end{equation*}
Ici, $V_0$ est simplement la valeur de l'option vendue sous la modèle de Black-Scholes. La
matrice \verb+phiPut+ représente quant à elle l'évaluation du delta de l'option à chaque
observation effectuée sur le prix du sous-jacent, pour chacun des scénarios, alors que
\verb+Delta+ est le changement de prix actualisés (à référence $t=0$) du sous-jacent entre
chacune des observations, ce qui représente effectivement une application directe du
rebalancement delta. 

\section*{Question 2}

Dans un premier temps nous adoptons le modèle de Black Scholes pour générer $N$
trajectoires possibles de prix selon l'équation suivante:
\begin{equation*}
  S_t = S_0 \exp\left(\left(\mu - \frac{\sigma^2}{2}\right)t + \sigma W_t\right),
\end{equation*}
où $W_t$ est un mouvement brownien.

La fonction \verb+generateBSPrice+ retourne alors une matrice $n\times N$ de $n$
observations et de $N$ trajectoire prix selon les divers paramètres nécessaires. Voici son
implantation:
\begin{equation*}
  \left.
    \begin{array}{l}
      \verb|function S = generateBSPrices(S0,mu,sigma,T,n,N)|\\
      \verb|% T: length of the observation|\\
      \verb|% n: number of observations|\\
      \verb|% N: number of trajectories|\\
      \verb||\\
      \verb|h = T/n;|\\
      \verb|mean = h*(mu - sigma^2/2);|\\
      \verb|vol = sqrt(h)*sigma;|\\
      \verb|X = normrnd(mean,vol,n,N);|\\
      \verb|X = [zeros(1,N); X];|\\
      \verb|S = S0*exp(cumsum(X));|\\
      \verb|end|\\
    \end{array}
  \right.
\end{equation*}
Nous employons ici le fait que les rendements sous l'hypothèse Black Scholes sont iid et
de loi 
\begin{equation*}
  R_k = h\left(\mu - \frac{\sigma^2}{2}\right) + \sigma\sqrt{h}Z,
\end{equation*}
où $Z\sim N(0,1)$ une normale centrée réduite et $h=T/n$.  

Pour mesurer l'efficacité de la couverture optimale, nous avons également mis au point une
fonction \verb+optHedging+, mais qui appelle la fonction compilée
\verb+Hedging_IID_MC2012+. Cette fonction prend comme entrée les mêmes paramètres que
\verb+hedging+, avec en plus un booléen \verb+gamma+ qui indique la nature du modèle
employée (variance gamma ou BMS):
\begin{equation*}
  \left.
    \begin{array}{l}
      \verb|function V = optHedging(S,K,r,T,mu,sigma,put,gamma)|\\
\verb|m = 2000;|\\
\verb|[n,N] = size(S);|\\
\verb|h = T/n;|\\
\verb|beta = diag(exp(-r*linspace(0,T,n)));|\\
\verb|S = beta*S;  |\\
\verb|Delta = S(2:end,:) - S(1:end-1,:);|\\
\verb||\\
\verb|nTraining = 10000;|\\
\verb|if ~gamma|\\
\verb|    % BMS |\\
\verb|    mean = h*(mu - r - sigma^2/2);|\\
\verb|    vol = sqrt(h)*sigma;|\\
\verb|    R = normrnd(mean,vol,nTraining,1);|\\
\verb|else|\\
\verb|    % Variance Gamma returns|\\
\verb|    mean = h*(mu - r - sigma^2/2);|\\
\verb|    V = gamrnd(h,1,nTraining,1);|\\
\verb|    Z = normrnd(0,1,nTraining,1);|\\
\verb|    G = sqrt(V).*Z;|\\
\verb|    R = mean + sigma*G;|\\
\verb|end|\\
\verb||\\
\verb|minS = min(min(S));|\\
\verb|maxS = max(max(S));|\\
\verb|[~,C,a,c1,~] = Hedging_IID_MC2012(R,T,K,r,n,put,minS,maxS,m);|\\
\verb|V = zeros(N,1);|\\
\verb|V0 = interpolation_1d(S(1,1),C(1,:)',minS,maxS);|\\
\verb||\\
\verb|for i=1:N|\\
\verb|    pi = V0;|\\
\verb||\\
\verb|    for k=1:n-1|\\
\verb|        ak = interpolation_1d(S(k,i),a(k,:)',minS,maxS);|\\
\verb|        phi = (ak - c1*pi)/S(k,i);|\\
\verb|        pi = pi + phi*Delta(k,i);|\\
\verb|    end|\\
\verb||\\
\verb|    V(i) = pi;|\\
\verb|end|\\
\verb|end|\\
    \end{array}
  \right.
\end{equation*}

Comme on peut le voir, cette fonction calcule dans un premier temps les paramètres
`hors-ligne' nécessaires à l'évaluation de la couverture optimale à partir de
\verb+nTraining=10000+ rendements iid, puis évalue les paramètres $C,a$ et $c_1$ en
utilisant une grille de 2000 points.

Enfin, on détermine récursivement la valeur du portefeuille de réplication à échéance à
partir du paramètre $\phi_k$ indiquant le nombre de positions à acheter suivant la formule 
\begin{equation*}
  \phi_k = \frac{a_k(\tilde S_{k-1}) - \hat c_1 \tilde \pi_{k-1}}{\tilde S_{k-1}}.
\end{equation*}
La valeur de $a_k(\tilde S_{k-1})$ est interpolée à l'aide de la fonction compilée
\verb+interpolation_1d+. 

\subsection*{Commentaires et résultats}

Les résultats statistiques suivants ont été générés à partir des paramètres suivants:
\begin{table}[h]
  \centering
  \begin{tabular}[h]{ll}
    % \toprule
    $r$ & $0.02$\\
    $S_0$ & $100$\\
    $\mu$ & $0.5$\\
    $T$ & $1$\\
    $K$ & $\{95,100,105\}$\\
    $n$ & $\{5,21,63,252\}$\\
    % \bottomrule
  \end{tabular}
  \caption{Paramètres de test}
  \label{tab:1}
\end{table}

La Table 2 présente les résultats pour les différentes combinaisons de $K$ et $n$ (cette
table a d'ailleurs été générée automatiquement à l'aide de la fonction
\verb+statsGeneration+). 
\begin{table}
\footnotesize{\begin{tabular}{lllllll}
\toprule
& \multicolumn{2}{c}{Moyenne} & \multicolumn{2}{c}{Médiane} & \multicolumn{2}{c}{Volatilité}\\
$(n,K)$& Delta & Optimal & Delta & Optimal & Delta & Optimal\\
\midrule
$(5,95)$ & \num{-2.001} & \num{-0.017} & \num{-1.802} & \num{-8.603e-05} & \num{2.014} & \num{0.433}\\
$(5,100)$ & \num{-2.531} & \num{-0.019} & \num{-2.310} & \num{-2.876e-04} & \num{2.468} & \num{0.627}\\
$(5,105)$ & \num{-3.076} & \num{-0.025} & \num{-2.908} & \num{-1.290e-03} & \num{2.962} & \num{0.894}\\
$(21,95)$ & \num{-0.434} & \num{3.206e-04} & \num{-0.401} & \num{-9.974e-05} & \num{0.763} & \num{0.211}\\
$(21,100)$ & \num{-0.545} & \num{3.503e-03} & \num{-0.528} & \num{-3.341e-04} & \num{0.943} & \num{0.279}\\
$(21,105)$ & \num{-0.689} & \num{-2.318e-04} & \num{-0.663} & \num{-3.163e-04} & \num{1.133} & \num{0.432}\\
$(63,95)$ & \num{-0.139} & \num{2.633e-03} & \num{-0.134} & \num{-7.088e-05} & \num{0.430} & \num{0.140}\\
$(63,100)$ & \num{-0.184} & \num{-1.801e-04} & \num{-0.176} & \num{-1.488e-04} & \num{0.515} & \num{0.163}\\
$(63,105)$ & \num{-0.229} & \num{5.520e-03} & \num{-0.218} & \num{-1.064e-04} & \num{0.613} & \num{0.185}\\
$(252,95)$ & \num{-0.033} & \num{-8.275e-04} & \num{-0.030} & \num{-1.115e-05} & \num{0.204} & \num{0.054}\\
$(252,100)$ & \num{-0.045} & \num{-7.916e-04} & \num{-0.046} & \num{-1.004e-04} & \num{0.253} & \num{0.085}\\
$(252,105)$ & \num{-0.057} & \num{-6.527e-04} & \num{-0.054} & \num{-1.238e-04} & \num{0.300} & \num{0.101}\\
\bottomrule
\end{tabular}
\\[3em]
\begin{tabular}{lllllllll}
\toprule
& \multicolumn{2}{c}{Asymétrie} & \multicolumn{2}{c}{Kurtose} & \multicolumn{2}{c}{Max.} & \multicolumn{2}{c}{Min.}\\
$(n,K)$& Delta & Optimal & Delta & Optimal & Delta & Optimal & Delta & Optimal\\
\midrule
$(5,95)$ & \num{-0.960} & \num{-18.298} & \num{6.493} & \num{550.450} & \num{5.967} & \num{3.561} & \num{-17.620} & \num{-18.459}\\
$(5,100)$ & \num{-0.726} & \num{-10.554} & \num{5.287} & \num{210.273} & \num{7.226} & \num{5.468} & \num{-19.847} & \num{-15.660}\\
$(5,105)$ & \num{-0.471} & \num{-5.883} & \num{4.702} & \num{83.997} & \num{7.424} & \num{7.121} & \num{-25.449} & \num{-15.443}\\
$(21,95)$ & \num{-0.238} & \num{-10.508} & \num{5.399} & \num{692.137} & \num{5.319} & \num{5.303} & \num{-5.447} & \num{-9.938}\\
$(21,100)$ & \num{-0.175} & \num{-0.515} & \num{4.866} & \num{96.207} & \num{5.370} & \num{5.337} & \num{-6.471} & \num{-5.509}\\
$(21,105)$ & \num{-0.159} & \num{-14.988} & \num{4.397} & \num{566.911} & \num{5.080} & \num{3.965} & \num{-9.099} & \num{-19.727}\\
$(63,95)$ & \num{-0.042} & \num{-5.376} & \num{6.787} & \num{452.351} & \num{2.879} & \num{3.045} & \num{-3.619} & \num{-5.012}\\
$(63,100)$ & \num{-0.193} & \num{0.390} & \num{6.128} & \num{212.722} & \num{2.601} & \num{4.191} & \num{-4.575} & \num{-4.233}\\
$(63,105)$ & \num{-0.156} & \num{1.797} & \num{5.291} & \num{134.494} & \num{3.701} & \num{3.784} & \num{-5.403} & \num{-4.660}\\
$(252,95)$ & \num{-0.122} & \num{-8.880} & \num{5.816} & \num{590.829} & \num{1.246} & \num{1.344} & \num{-1.562} & \num{-2.237}\\
$(252,100)$ & \num{0.115} & \num{-2.418} & \num{5.077} & \num{249.390} & \num{1.812} & \num{1.722} & \num{-1.410} & \num{-2.524}\\
$(252,105)$ & \num{-0.100} & \num{-3.130} & \num{5.660} & \num{217.167} & \num{2.026} & \num{1.646} & \num{-2.600} & \num{-3.328}\\
\bottomrule
\end{tabular}}
\caption{Statistiques pour stratégies de couvertures pour des prix suivant une loi
  log-normale (BMS)}
\end{table}

Enfin, la figure 1 présente les évaluations de densité d'erreur pour chacune des
combinaisons de $K$ et $n$. Ces figures ont été générées par le script
\verb+figureGeneration+, à l'aide de la fonction \verb+ksdensity+ de Matlab. A noter que
cette figure est également disponible individuellement en pièce jointe à ce document
(\verb+BMSHedging.pdf+). 
\begin{figure}
  \centering
  \includegraphics[angle=90,scale=0.5]{BMSHedging}
  \caption{Densité de probabilité d'erreur de couverture avec rendements BMS}
  \label{fig:1}
\end{figure}

On observe que l’erreur de couverture moyenne est supérieure lorsqu’on utilise une
couverture par delta que lorsqu’on utilise une couverture optimale. Par delta, l’erreur
est très grande lorsqu’on effectue seulement 5 rebalancements en un an. En effet, lorsque
$K=100$ et $S = 100$ (option à la monnaie), la valeur du put est de 6 à 7\$ environ (selon
les simulations). Une erreur de 2 ou 3 dollars représente une erreur de près de 50\%, ce
qui est énorme. Ceci est cohérent avec les constats empiriques connus, selon lesquels le
`delta-hedging' nécessite de fréquents rebalancements (ce qui est coûteux en termes de
frais de transactions) pour être efficace. Et de fait, l’erreur diminue avec les
rebalancements, jusqu’à atteindre – 0.045 (cas de l’option à la monnaie) lorsqu’on
rebalance chaque jour. C’est une erreur inférieure à 1\%, ce qui semble raisonnable.

De ce point de vue, la couverture optimale semble beaucoup plus efficace et s’annonce
moins coûteuse puisqu’elle permet une erreur de couverture inférieure à 0.5\%, et ce, avec
seulement 5 rebalancements. Augmenter le nombre de rebalancements rend l’erreur quasi
infime.

Si maintenant, pour un nombre de rebalancements donné, on compare les 3 cas: en jeu
($K= 95$), à la monnaie ($K = 100$) et hors-jeu (K = 105), l’erreur grossit à mesure qu’on
sort de la monnaie, quand on suit une stratégie basée sur le delta. Elle est d’environ
50\% plus grande à $K = 105$ qu’à $K = 95$ (\num{-0.689} contre \num{-0.434} par exemple,
dans le cas de 63 rebalancements par an, donc un rebalancement tous les 4 jours
environ). Ceci est possiblement lié au fait que le delta d’un put passe de \num{-1} à 0 à
mesure qu’on sort de la monnaie. Le lien entre le sous-jacent et l’option se distend donc
et, logiquement, on peut s’attendre à ce qu’une stratégie de couverture basée sur ce lien
soit de moins en moins efficace.

Lorsqu’on effectue une couverture de type optimal, le phénomène s’observe seulement
lorsque les rebalancements sont rares ($n = 5$). Dès que les rebalancements s’accélèrent,
le phénomène disparaît, ce qui confirme l’efficacité de cette stratégie.

Toutefois, si l’on observe les écarts-types, ils augmentent à mesure qu’on sort de la
monnaie, y compris pour ce type de stratégie de type optimal. Il semble donc
intrinsèquement plus difficile, et ce, quelle que soit la stratégie de couverture choisie,
de couvrir une option hors de la monnaie qu’une option en jeu.

L’écart-type (\textit{volatilité} dans le Tableau) de l’erreur est nettement supérieur
sous stratégie delta que sous couverture de type optimale. Ceci confirme la supériorité de
la stratégie de type optimal. Ceci confirme également la nécessité des rebalancements
fréquents si l’on se couvre par stratégie de couverture de type delta. En effet, avec 5
rebalancements, non seulement l’erreur moyenne est élevée comme on a vu plus haut, mais
l’écart-type lui-même l’est. En moyenne, on a une erreur de 2.5\$ sur 6\$ à 7\$ environ;
mais en plus, en moyenne, l’erreur s’écarte de 2.468\$ de cette moyenne. Le risque est
donc élevé d’être à 70\% ou 80\% découvert. Encore une fois, sans rebalancement fréquent, il
faudrait être bien téméraire pour se couvrir par delta. Même à 21 rebalancements annuels,
soit environ 2 par mois, l’écart-type de l’erreur est d’environ 1\$ soit environ 15\% de la
valeur de l’option (option à la monnaie).

Le fait que l’erreur de couverture diminue à mesure que la fréquence de rebalancement
augmente se reflète également dans l’erreur-type (RMSE) qui diminue à mesure que n
augmente.

Trois statistiques nous obligent à apporter un léger bémol dans une comparaison couverture
par delta vs. couverture de type optimal: aplatissement, asymétrie et minimum. Pour ces
trois statistiques, la couverture dite optimale présente de moins bonnes
caractéristiques: davantage leptokurtique, cette couverture a des queues plus épaisses et
plus grand risque d’événements, certes rares, mais très négatif. Ceci est confirmé par une
asymétrie négative et des minima plus grands (en valeur absolue). Ceci se remarque
également visuellement dans les graphiques ci-dessous: graphiques en dents-de-scie de la
fonction de densité de l’erreur de couverture dite optimale (par exemple, $n = 63, K =100$),
avec une `dent' à gauche qui dépasse de la cloche normale; ou longueur des queues de
couleur rouge dans nombre de graphiques. 

Toutefois, ces statistiques sont moins décisives que la moyenne et l’écart-type, et la
couverture de type optimal semble définitivement plus performante, même si elle n’est pas
absolument sans risque. 


\section*{Question 3}

Dans cette question, nous reprenons la même analyse qu'à la précédente section, en
employant cette fois un modèle variance gamma d'évolution des prix. Ceux-ci obéissent
alors à la loi
\begin{equation*}
  S_t = S_0\exp\left(\left(\mu - \frac{\sigma^2}{2}\right)t + \sigma G_t\right),
\end{equation*}
où $G_t-G_s$ suit une loi $\sqrt{V}Z$, avec $Z\sim N(0,1)$ et $V\sim\gam(t-s,1)$,
indépendants l'un à l'autre. 

Selon Madan et al. (1998), 
\begin{equation*}
  R_k = \left(\mu - \frac{\sigma^2}{2}\right) \frac{T}{n} +
  \frac{\sigma}{\sqrt{2\alpha}}(Y_{1,k} - Y_{2,k})
\end{equation*}
est l'expression générale pour les rendements excédentaires iid. On alors
\begin{equation*}
  R_k = \left(\mu - \frac{\sigma^2}{2}\right)\frac{T}{n} + \sigma\sqrt{V}Z.
\end{equation*}
Si $\alpha=1$, alors on voit que les rendements annuels suivent une loi de Laplace, ce qui
revient à une différence de deux variables aléatoires de loi exponentielle. On en conclut
que $G_1$ suit une loi exponentielle. 

Algorithmiquement, la fonction \verb+generateGammaPrices+ produit une matrice de prix de
lois variance gamma:
\begin{equation*}
  \left.
    \begin{array}{l}
      \verb|function S = generateGammaPrices(S0,mu,sigma,T,n,N)|\\
      \verb|h = T/n;|\\
      \verb|mean = h*(mu - sigma^2/2);|\\
      \verb|V = gamrnd(h,1,n,N);|\\
      \verb|Z = normrnd(0,1,n,N);|\\
      \verb|G = sqrt(V).*Z;|\\
      \verb|X = mean + sigma*G;|\\
      \verb|X = [zeros(1,N); X];|\\
      \verb|S = S0*exp(cumsum(X));|\\
      \verb|end|\\
    \end{array}
  \right.
\end{equation*}

\subsection*{Commentaires et résultats}

Les statistiques suivantes sont générés à partir des paramètres suivants:
\begin{table}[h]
  \centering
  \begin{tabular}[h]{ll}
    % \toprule
    $r$ & $0.02$\\
    $S_0$ & $100$\\
    $\mu$ & $0.3$\\
    $T$ & $1$\\
    $K$ & $\{95,100,105\}$\\
    $n$ & $\{5,21,63,252\}$\\
    % \bottomrule
  \end{tabular}
  \caption{Paramètres de test pour les prix Variance Gamma}
  \label{tab:1}
\end{table}

La Table 4 présente les diverses statistiques calculées pour chacune des combinaisons de
paramètres, de la même facon à la Figure 2 en ce qui concerne la densité des erreurs de
couverture. 

\begin{table}
\footnotesize{\begin{tabular}{lllllll}
\toprule
& \multicolumn{2}{c}{Moyenne} & \multicolumn{2}{c}{Médiane} & \multicolumn{2}{c}{Volatilité}\\
$(n,K)$& Delta & Optimal & Delta & Optimal & Delta & Optimal\\
\midrule
$(5,95)$ & \num{-1.980} & \num{0.061} & \num{-1.759} & \num{-0.018} & \num{2.018} & \num{0.532}\\
$(5,100)$ & \num{-2.550} & \num{0.105} & \num{-2.398} & \num{-0.013} & \num{2.487} & \num{0.771}\\
$(5,105)$ & \num{-3.101} & \num{0.163} & \num{-2.938} & \num{-0.023} & \num{2.903} & \num{1.019}\\
$(21,95)$ & \num{-0.428} & \num{0.042} & \num{-0.403} & \num{-9.709e-03} & \num{0.741} & \num{0.326}\\
$(21,100)$ & \num{-0.560} & \num{0.080} & \num{-0.523} & \num{-0.014} & \num{0.951} & \num{0.500}\\
$(21,105)$ & \num{-0.697} & \num{0.036} & \num{-0.671} & \num{-6.536e-03} & \num{1.110} & \num{0.550}\\
$(63,95)$ & \num{-0.138} & \num{0.016} & \num{-0.133} & \num{-8.556e-03} & \num{0.420} & \num{0.264}\\
$(63,100)$ & \num{-0.181} & \num{0.055} & \num{-0.177} & \num{-0.040} & \num{0.520} & \num{0.608}\\
$(63,105)$ & \num{-0.238} & \num{0.047} & \num{-0.235} & \num{-0.037} & \num{0.609} & \num{0.596}\\
$(252,95)$ & \num{-0.035} & \num{0.139} & \num{-0.033} & \num{-0.052} & \num{0.202} & \num{0.735}\\
$(252,100)$ & \num{-0.047} & \num{0.091} & \num{-0.047} & \num{-0.014} & \num{0.255} & \num{0.523}\\
$(252,105)$ & \num{-0.052} & \num{-8.376e-03} & \num{-0.048} & \num{-0.021} & \num{0.307} & \num{0.470}\\
\bottomrule
\end{tabular}
\\[3em]
\begin{tabular}{lllllll}
\toprule
& \multicolumn{2}{c}{Asymétrie} & \multicolumn{2}{c}{Kurtose} & \multicolumn{2}{c}{Max.}\\
$(n,K)$& Delta & Optimal & Delta & Optimal & Delta & Optimal\\
\midrule
$(5,95)$ & \num{-0.890} & \num{0.310} & \num{6.319} & \num{35.458} & \num{6.192} & \num{3.938}\\
$(5,100)$ & \num{-0.550} & \num{-1.860} & \num{4.809} & \num{42.882} & \num{7.048} & \num{4.387}\\
$(5,105)$ & \num{-0.456} & \num{-0.633} & \num{4.565} & \num{24.264} & \num{7.159} & \num{10.790}\\
$(21,95)$ & \num{-0.387} & \num{2.811} & \num{5.619} & \num{66.646} & \num{2.785} & \num{4.631}\\
$(21,100)$ & \num{-0.491} & \num{0.812} & \num{6.583} & \num{58.340} & \num{3.957} & \num{5.617}\\
$(21,105)$ & \num{-0.327} & \num{-9.183} & \num{4.972} & \num{189.485} & \num{4.176} & \num{4.265}\\
$(63,95)$ & \num{-0.088} & \num{5.397} & \num{5.850} & \num{183.683} & \num{3.295} & \num{6.762}\\
$(63,100)$ & \num{-0.183} & \num{5.258} & \num{5.434} & \num{117.104} & \num{3.430} & \num{14.833}\\
$(63,105)$ & \num{-0.193} & \num{-4.131} & \num{5.195} & \num{212.556} & \num{3.404} & \num{7.808}\\
$(252,95)$ & \num{-0.160} & \num{5.758} & \num{6.758} & \num{57.476} & \num{1.217} & \num{13.677}\\
$(252,100)$ & \num{-0.114} & \num{5.203} & \num{5.330} & \num{99.692} & \num{1.436} & \num{12.269}\\
$(252,105)$ & \num{-0.136} & \num{-2.481} & \num{5.309} & \num{228.089} & \num{1.704} & \num{9.265}\\
\bottomrule
\end{tabular}
\\[3em]
\begin{tabular}{lllllll}
\toprule
& \multicolumn{2}{c}{Min.} & \multicolumn{2}{c}{Var 99\%} & \multicolumn{2}{c}{RMSE}\\
$(n,K)$& Delta & Optimal & Delta & Optimal & Delta & Optimal\\
\midrule
$(5,95)$ & \num{-21.386} & \num{-11.061} & \num{2.184} & \num{2.216} & \num{2.827} & \num{0.535}\\
$(5,100)$ & \num{-16.573} & \num{-13.822} & \num{3.290} & \num{2.841} & \num{3.562} & \num{0.778}\\
$(5,105)$ & \num{-20.532} & \num{-12.986} & \num{3.857} & \num{3.640} & \num{4.247} & \num{1.032}\\
$(21,95)$ & \num{-7.318} & \num{-5.849} & \num{1.357} & \num{1.509} & \num{0.856} & \num{0.328}\\
$(21,100)$ & \num{-9.968} & \num{-9.488} & \num{1.692} & \num{2.107} & \num{1.104} & \num{0.506}\\
$(21,105)$ & \num{-8.536} & \num{-14.334} & \num{1.935} & \num{1.547} & \num{1.311} & \num{0.551}\\
$(63,95)$ & \num{-3.005} & \num{-5.838} & \num{0.915} & \num{0.914} & \num{0.442} & \num{0.264}\\
$(63,100)$ & \num{-3.425} & \num{-14.613} & \num{1.093} & \num{2.734} & \num{0.551} & \num{0.610}\\
$(63,105)$ & \num{-4.372} & \num{-20.878} & \num{1.305} & \num{2.265} & \num{0.654} & \num{0.598}\\
$(252,95)$ & \num{-2.303} & \num{-0.768} & \num{0.490} & \num{3.294} & \num{0.205} & \num{0.748}\\
$(252,100)$ & \num{-1.892} & \num{-11.144} & \num{0.601} & \num{2.365} & \num{0.259} & \num{0.530}\\
$(252,105)$ & \num{-2.409} & \num{-16.614} & \num{0.730} & \num{1.773} & \num{0.311} & \num{0.470}\\
\bottomrule
\end{tabular}}
\caption{Statistiques pour stratégies de couvertures pour des prix suivant une loi
  variance gamma}
\end{table}

\begin{figure}
  \centering
  \includegraphics[angle=90,scale=0.5]{GammaHedging}
  \caption{Densité de probabilité d'erreur de couverture avec rendements BMS}
  \label{fig:1}
\end{figure}

Cette fois, la méthode la plus efficace est disputée. En effet, au niveau de la moyenne,
on observe clairement de meilleures performances du côté du rebalancement optimal qui
conserve, en général, peu importe le nombre de rebalancements, des moyennes de l'ordre de
\num{e-2}, alors que la méthode delta semble stagner à un ordre de grandeur plus élevé. 

Au niveau de la médiane, qui caractérise la valeur qui est la plus vraisemblable, la
méthode optimale offre une relative stabilité par rapport au nombre de rebalancements,
alors que la méthode delta a tendance à offrir des performances dégradées à mesure que le
nombre de rebalancement augmente.

Pour ce qui est de l'écart-type des résultats, ie. la volatilité, encore une fois, la
méthode optimale offre en général des valeurs plus serrées que la méthode delta. Aucune
des deux méthode par contre ne semble bénéficier ou souffrir d'une fréquence faible ou
élevée de rebalancements. 

L'asymétrie est assez frappante chez les deux types de méthodes, car on remarque
graphiquement une asymétrie négative. De plus, comme on le voit par la statistique Min.,
cette asymétrie semble aller chercher des valeurs encore plus faibles lorsque l'option est
hors de la monnaie, ie. $K>S_0$. 

La mesure de kurtose est également intéressante, car elle trahit une distribution très
étalée de l'erreur, qui est d'ailleurs souligné par la mesure de la VaR, qui indique
effectivement que le premier percentile est extrêmement élevé, particulièrement pour la
méthode delta qui indique une VaR deux fois plus élevée que celle de la méthode optimale.

Donc, bien qu'aucune des deux méthodes ne semble particulièrement performante, la méthode
optimale demeure tout de même la méthode qui serait à privilégier dans un contexte où les
rendements sont suspectés ne pas être de nature brownienne. 


\section*{Conclusion}

Si l’on suppose que les rendements sont gaussiens et suivent une loi lognormale (voir
Question 2), une couverture de type optimal couvre nettement mieux qu’une couverture de
type delta. Elle offre une couverture très bonne (erreur moyenne et écart-type des erreurs
minimes) même avec une fréquence de rebalancement très faible. Elle n’est toutefois pas
sûre à 100\% et présente même de plus longues queues et des minima plus petits (plus
proche de moins l’infini) que la couverture par delta.

Même au niveau d'un modèle variance gamma, le rebalancement optimal demeure la `moins pire
des options'. Mais il est essentiel de comprendre que ni l'une ni l'autre n'offre de
performance vraiment convaincante, bien que leur moyenne semble tendre plus ou moins vers
0. Toutefois, ce qu'on remarque c'est que des rebalancements fréquents n'offrent pas
nécessairement une protection supplémentaire et qu'en ce sens, dans un contexte où des
frais aux transactions sont appliqués, des rebalancements moins fréquents pourraient être
plus avantageux pour l'émissaire d'une option de vente. 


\end{document}
%%% Local Variables:
%%% mode: latex
%%% TeX-master: t
%%% End:
